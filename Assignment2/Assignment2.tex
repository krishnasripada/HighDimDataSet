\documentclass[10pt]{article}
\usepackage{amsmath}
\usepackage{mathtools}
\DeclarePairedDelimiter{\abs}{\lvert}{\rvert}
\usepackage[hidelinks]{hyperref}
\usepackage{amssymb}
\usepackage{tikz}
\usepackage{caption}
\usepackage{graphicx}
\usepackage[T1]{fontenc}
\graphicspath{{.}}
\usepackage{listings}
\usepackage{verbatim}
\usepackage{floatrow}
\usepackage{bigints}
\lstset{
language=[LaTeX]TeX,
backgroundcolor=\color{gray!25},
basicstyle=\ttfamily,
columns=flexible,
breaklines=true
}
\captionsetup{labelsep=space,justification=centering,singlelinecheck=off}
\reversemarginpar
\usepackage[paper=a4paper,
            %includefoot, % Uncomment to put page number above margin
            marginparwidth=10mm,      % Length of section titles
            marginparsep=0.8mm,       % Space between titles and text
            margin=11mm,              % 25mm margins
            includemp]{geometry}

\begin{document}
\section*{}
\begin{flushleft}
Name: Krishna Chaitanya Sripada\\
\end{flushleft}
\section*{Ans 1}
\begin{flushleft}
To prove $O(n)$ is a group when equipped with the matrix multiplication, we need to prove that the 4 points mentioned in Definition 1 hold good.\\
\vspace{0.5em}
\textbf{Point 1:} We need to prove that on multiplying two orthogonal matrices in $O(n)$, we get back an orthogonal matrix thus meaning the resultant matrix is also in $O(n)$.\\
Let us consider two orthogonal matrices G and H, then,\\
\vspace{0.5em}
$(G G^{T}) (H H^{T}) = (G H) (G^{T} H ^{T}) = (H G) (H G)^{T} = I$. This means that the resultant matrix is in the group $O(n)$.\\
\vspace{0.5em}
\textbf{Proof by example:} $A_{n \times n}$ is an orthogonal matrix if, $A A^{T} = A^{T} A = I$. Consider the multiplication of following orthogonal matrices,\\
$$
\begin{bmatrix} 
-1 & 0 & 0 \\
0 & -1 & 0 \\
0 & 0 & -1 
\end{bmatrix}
\begin{bmatrix} 
0 & -1 & 0 \\
1 & 0 & 0 \\
0 & 0 & -1 
\end{bmatrix}
= \begin{bmatrix} 
0 & 1 & 0 \\
-1 & 0 & 0 \\
0 & 0 & 1 
\end{bmatrix}
 = A
$$
\\
The transpose of this matrix is ,\\
$$
A^{T} =
\begin{bmatrix} 
0 & -1 & 0 \\
1 & 0 & 0 \\
0 & 0 & 1 
\end{bmatrix}
$$
\\
Therefore, 
$$
A A^{T} =
\begin{bmatrix} 
0 & 1 & 0 \\
-1 & 0 & 0 \\
0 & 0 & 1 
\end{bmatrix}
\begin{bmatrix} 
0 & -1 & 0 \\
1 & 0 & 0 \\
0 & 0 & 1 
\end{bmatrix}
= 
\begin{bmatrix} 
1 & 0 & 0 \\
0 & 1 & 0 \\
0 & 0 & 1 
\end{bmatrix}
 = I
$$
\\
Also,
$$
A^{T} A=
\begin{bmatrix} 
0 & -1 & 0 \\
1 & 0 & 0 \\
0 & 0 & 1 
\end{bmatrix}
\begin{bmatrix} 
0 & 1 & 0 \\
-1 & 0 & 0 \\
0 & 0 & 1 
\end{bmatrix}
= 
\begin{bmatrix} 
1 & 0 & 0 \\
0 & 1 & 0 \\
0 & 0 & 1 
\end{bmatrix}
 = I
$$
\\
The resultant matrix also belongs to group $O(n)$.\\
\vspace{0.5em}
\textbf{Point 2:}  We need to prove that matrix multiplication of 3 matrices that belong to $O(n)$ is associative.\\
Let us consider three orthogonal matrices F, G and H, then,\\
\vspace{0.5em}
$\sum_{p} \sum_{q} F_{ip} G_{pq} H_{qj} = \sum_{p} F_{ip} (\sum_{q} G_{pq} H_{qj}) = \sum_{p} F_{ip} (GH)_{pj} = F(GH)$\\
\vspace{0.5em}
$\sum_{p} \sum_{q} F_{ip} G_{pq} H_{qj} = \sum_{q} (\sum_{p} F_{ip} G_{pq}) H_{qj} = \sum_{q} (FG)_{iq} H_{qj} = (FG)H$\\
\vspace{0.5em}
\textbf{Point 3:} 
Let the element that belongs to the group $O(n)$ be an identity matrix (as Identity matrix is also orthogonal), then for another orthogonal matrix G $\in O(n)$, we get,\\
\vspace{0.5em}
$G (I) = I (G) = G$. This is because matrix multiplication of an identity matrix and any other matrix is commutative in nature and returns the original matrix. \\
\vspace{0.5em}
\textbf{Point 4:}
We need to prove that the multiplication of an inverse and the matrix itself is an identity matrix which also belongs to the group $O(n)$.\\
Since, inverse of an orthogonal matrix is also orthogonal, it follows the Point 1 stated above where we check for the multiplication of two orthogonal matrices. \\
In this case, we get,\\
$G (G^{-1}) = I$ and $G^{-1} (G) = I$. Thus we can prove that the identity matrix also belongs to $O(n)$, we can say an inverse exists in $O(n)$.
\end{flushleft}
\section*{Ans 2}
\begin{flushleft}
Since $U \in O(n)$, this means that U is in the group of orthogonal matrices, so for any matrix M that belongs to this group, we know that, \\
\vspace{0.5em}
$M^{T} = M^{-1}$ (The transpose of M is the inverse of M). \\
\vspace{0.5em}
Also, $M M^{-1} = I$\\
\vspace{0.5em}
Therefore, \\
\vspace{0.5em}
$M M^{T} = I$ \\
\vspace{0.5em}
Now taking a determinant on both sides,
\vspace{0.5em}
$det ( M M^{T}) = det I$\\
\vspace{0.5em}
We know that $det M * det M^{T} = 1$ (since $det I = 1$).\\
\vspace{0.5em}
Now $det M * det M = 1$ (since $det M^{T} = det M$)\\
\vspace{0.5em}
Meaning $(det M)^{2} = 1$\\
\vspace{0.5em}
Therefore, det(M) = $\pm$ 1.
\end{flushleft}
\section*{Ans 3}
\begin{flushleft}
To prove $SO(n)$ is a subgroup of $O(n)$, we need to prove that the 2 points mentioned in Definition 2 hold good.\\
\vspace{0.5em}
\textbf{Point 1:} To prove this, let us consider two orthogonal matrices G, H $\in$ $SO(n)$, this means that the product of two orthogonal matrices should belong to the $SO(n)$.\\
\vspace{0.5em}
\textbf{Proof by example:} Consider the multiplication of the following orthogonal matrices that below to $SO(n)$. \\
$$
\begin{bmatrix} 
1 & 0 & 0 \\
0 & -1 & 0 \\
0 & 0 & -1 
\end{bmatrix} 
\begin{bmatrix} 
-1 & 0 & 0 \\
0 & 1 & 0 \\
0 & 0 & -1 
\end{bmatrix} 
= 
\begin{bmatrix} 
-1 & 0 & 0 \\
0 & -1 & 0 \\
0 & 0 & 1 
\end{bmatrix} 
$$
\\
The determinant of the resultant matrix  = 1 which belongs to $SO(n)$ which is an special orthogonal matrix. Thus the resultant matrix also belongs to $SO(n)$.\\
\vspace{0.5em}
\textbf{Point 2: } We need to prove that the inverse of a matrix and the multiplication of the same matrix is an identity matrix. \\
\vspace{0.5em}
Since, inverse of an orthogonal matrix is also orthogonal, it follows the Point 1 stated above where we check for the multiplication of two orthogonal matrices. \\
In this case, we get,\\
$G (G^{-1}) = I$ and $G^{-1} (G) = I$. Thus we can prove that the identity matrix also belongs to $SO(n)$, we can say an inverse exists in $SO(n)$.
\end{flushleft}
\section*{Ans 4}
\begin{flushleft}
Given that $$
e_{1} = 
\begin{bmatrix} 
1 \\
0 \\
\vdots \\
0 
\end{bmatrix} 
$$
and $\mathcal{G} = \{U \in SO(n), U e_{1} = e_{1}\}$\\

\end{flushleft}
\end{document}